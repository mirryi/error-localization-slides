\newcommand{\entails}{\ensuremath{\goodcolor{\colorOkSideJudge}{\vdash}}}
\newcommand{\syn}{\ensuremath{\goodcolor{\colorOkSideJudge}{\Rightarrow}}}
\newcommand{\ana}{\ensuremath{\goodcolor{\colorOkSideJudge}{\Leftarrow}}}

\newcommand{\ctxSyn}[3]{\ensuremath{#1 \entails #2 \syn #3}}
\newcommand{\ctxAna}[3]{\ensuremath{#1 \entails #2 \ana #3}}

\begin{frame}[fragile]{Marked lambda calculus: a tutorial}
  \textbf{Start:} a small gradually typed lambda calculus
  %
  \[\begin{array}{rcl}
    \TMV  & \Coloneqq & \TUnknown \mid \TNum \mid \TBool \mid \TArrow{\TMV}{\TMV} \\
    \EMV  & \Coloneqq & x \mid \ELam{x}{\TMV}{\EMV} \mid \EAp{\EMV}{\EMV}
            \mid           \ENumMV \mid \ETrue \mid \EFalse
  \end{array}\]

  \pause

  \hphantom{\textbf{Start:} }with a standard bidirectional type system
  %
  \pause
  \[\begin{array}{rl}
    \ctxSyn{\ctx}{\EMV}{\TMV} & \textsc{\small(synthesis)} \\ \pause
    \ctxAna{\ctx}{\EMV}{\TMV} & \textsc{\small(analysis)}
  \end{array}\]
\end{frame}

\begin{frame}[fragile]{Typing variable occurrences}
  Synthesizing the type of a variable is standard:
  \[%
    \inferrule[SVar]{
      \uncover<2->{\colorbox{highlight}{\inCtx{\ctx}{x}{\TMV}}}
    }{
      \ctxSyn{\ctx}{x}{\TMV}
    }
  \]%

  \pause
  \pause
  But what if $\notInCtx{\ctx}{x}$?
\end{frame}

\begin{frame}[fragile]
  How do we handle this failure case?

  \begin{lstlisting}[style=OCaml, escapechar=@]
let rec syn ctx e =
  match e with
    Var x ->
      match Ctx.lookup ctx x with
        Some ty -> @\only<1-3>{ty}\only<4>{\goodcolor{PineGreen}{Ok(ty)}}@
        None    -> @\goodcolor{red}{\only<1>{???}\only<2>{assert false}\only<3>{failwith (x ++ " is unbound")}\only<4>{Error(UnboundError(...))}}@
    ...
  \end{lstlisting}

  \only<1>{\phantom{Option 1}}
  \only<2>{Option 1: don't be ridiculous, that's impossible}
  \only<3>{Option 2: uh oh, not my problem}
  \only<4>{Option 3: uh oh, not my problem but \emph{gracefully}}
\end{frame}

\begin{frame}[fragile]
  We've \emph{localized} the error: \pause ``this occurrence of $x$ is unbound!'' \\[1em]

  \pause
  How can we \emph{recover}? \\[1em]

  \pause
  \textbf{Solution.} $\TUnknown$ \pause \hspace{1em} $\leftarrow$ unknown type
\end{frame}

\begin{frame}[fragile]{From type checking to marking}
  \textbf{Idea.} Augment the type checking process with \emph{marking}! \\[1em]

  \pause
  \begin{itemize}
    \item localize and report the error as a \emph{mark}
      \pause
      \begin{itemize}
        \item compiler messages

          \pause
        \item editor decorations
      \end{itemize}

      \pause
    \item use the unknown type to encapsulate missing type information
  \end{itemize}
\end{frame}

\begin{frame}[fragile]{A two-layer calculus}
  Supplement the \emph{unmarked} surface syntax

  \[\begin{array}{rcl}
    \TMV  & \Coloneqq & \TUnknown \mid \TNum \mid \TBool \mid \TArrow{\TMV}{\TMV} \\
    \EMV  & \Coloneqq & x \mid \ELam{x}{\TMV}{\EMV} \mid \EAp{\EMV}{\EMV}
            \mid           \ENumMV \mid \ETrue \mid \EFalse
            \hphantom{\ \mid           \ECFree{x}} \\
  \end{array}\]

  \pause
  into a \emph{marked} one that contains \emph{error marks}:
  \[\begin{array}{rcl}
    \ECMV & \Coloneqq & x \mid \ECLam{x}{\TMV}{\ECMV} \mid \ECAp{\ECMV}{\ECMV}
            \mid           \ECNumMV \mid \ECTrue \mid \ECFalse
            \mid           \ECFree{x}
  \end{array}\]

  \pause
  $\ECFree{x}$ is a \emph{marked term} denoting a free occurrence of $x$
\end{frame}

\begin{frame}[fragile]
  Extend the original typing judgments
  \[\begin{array}{rl}
    \ctxSyn{\ctx}{\EMV}{\TMV} & \textsc{\small(synthesis)} \\
    \ctxAna{\ctx}{\EMV}{\TMV} & \textsc{\small(analysis)}
  \end{array}\]

  \pause
  into the bidirectional \textbf{marking judgments}:

  \pause
  \[\begin{array}{rl}
    \ctxSynFixedInto{\ctx}{\EMV}{\ECMV}{\TMV} & \textsc{\small(marking synthesis)} \\ \pause
    \ctxAnaFixedInto{\ctx}{\EMV}{\ECMV}{\TMV} & \textsc{\small(marking analysis)}
  \end{array}\]
\end{frame}

\begin{frame}[fragile]{Marking free variables}
  One typing rule for variables 
  \[%
    \inferrule[SVar]{
      \inCtx{\ctx}{x}{\TMV}
    }{
      \ctxSyn{\ctx}{x}{\TMV}
    }
  \]%

  becomes two synthetic marking rules: 
  \begin{mathpar}
    \uncover<2->{
    \inferrule[MKSVar]{
      \inCtx{\ctx}{x}{\TMV}
    }{
      \ctxSynFixedInto{\ctx}{x}{x}{\TMV}
    }
    }

    \uncover<3->{
    \inferrule[MKSFree]{
      \notInCtx{\ctx}{x}
    }{
      \uncover<3->{\ctxSynFixedInto{\ctx}{x}{\uncover<4->{\ECFree{x}}}{\uncover<5->{\TUnknown}}}
    }
    }
  \end{mathpar}
\end{frame}

\begin{frame}[fragile]{A total marking}
  \begin{emphbox}{Guiding principle (totality)}
    These typing semantics should describe \\
    \emph{all syntactically well-formed programs}.
  \end{emphbox}
  %
  \pause
  \begin{center}
    $\bm{\Downarrow}$
  \end{center}

  \begin{emphbox}{Theorem 2.1. Marking Totality}
    \pause
    For all $\ctx$ and $\EMV$,
      $\exists \ECMV, \TMV$
        s.t. $\ctxSynFixedInto{\ctx}{\EMV}{\ECMV}{\TMV}$. \\
    For all $\ctx$, $\EMV$, and $\TMV$,
      $\exists \ECMV$
        s.t. $\ctxAnaFixedInto{\ctx}{\EMV}{\ECMV}{\TMV}$.
  \end{emphbox}
\end{frame}

\begin{frame}[fragile]{Marking function application}
  \textbf{Example.} The synthetic typing rule for function application:
  %
  \begin{mathpar}
    \inferrule[SAp]{
      \ctxSyn{\ctx}{\EMV_1}{\TMV} \\
      \matchedArrow{\TMV}{\TMV_1}{\TMV_2} \\
      \ctxAna{\ctx}{\EMV_2}{\TMV_1}
    }{
      \ctxSyn{\ctx}{\EAp{\EMV_1}{\EMV_2}}{\TMV_2}
    } \\

    \inferrule[TMAUnknown]{ }{
      \matchedArrow{\TUnknown}{\TUnknown}{\TUnknown}
    }

    \inferrule[TMAArr]{ }{
      \matchedArrow{\TArrow{\TMV_1}{\TMV_2}}{\TMV_1}{\TMV_2}
    }
  \end{mathpar}
\end{frame}

\begin{frame}[fragile]
  \ldots\ becomes the synthetic marking rule:
  \[%  
    \uncover<2->{
      \inferrule[MKSAp]{
      \uncover<3->{\ctxSynFixedInto{\ctx}{\EMV_1}{\ECMV_1}{\TMV}} \\
      \uncover<4->{\matchedArrow{\TMV}{\TMV_1}{\TMV_2}} \\
      \uncover<5->{\ctxAnaFixedInto{\ctx}{\EMV_2}{\ECMV_2}{\TMV_1}} \\
    }{
      \ctxSynFixedInto{\ctx}{\EAp{\EMV_1}{\EMV_2}}{\uncover<6->{\ECAp{\ECMV_1}{\ECMV_2}}}{\uncover<7->{\TMV_2}}
    }
    }
  \]%

  \uncover<8->{
    But what if $\TMV$ doesn't match an arrow type, $\notMatchedArrow{\TMV}$, \\
    \ie, $\EMV_1$ can't be applied as a function, \eg, in $\EAp{\ENum{5}}{\ETrue}$?
  }
\end{frame}

\begin{frame}[fragile]
  Introduce the mark $\ECApSynNonMatchedArrow{\ECMV_1}{\ECMV_2}$ for this error:
  \[%
    \ECMV \Coloneqq \cdots \mid \ECFree{x} \mid \only<1->{\ECApSynNonMatchedArrow{\ECMV}{\ECMV}}
  \]%
  \uncover<2->{
  \begin{mathpar}
    \inferrule[MKSAp1]{
      \ctxSynFixedInto{\ctx}{\EMV_1}{\ECMV_1}{\TMV} \\
      \matchedArrow{\TMV}{\TMV_1}{\TMV_2} \\
      \ctxAnaFixedInto{\ctx}{\EMV_2}{\ECMV_2}{\TMV_1} \\
    }{
      \ctxSynFixedInto{\ctx}{\EAp{\EMV_1}{\EMV_2}}{\ECAp{\ECMV_1}{\ECMV_2}}{\TMV_2}
    }

    \uncover<3->{
      \inferrule[MKSAp2]{
        \ctxSynFixedInto{\ctx}{\EMV_1}{\ECMV_1}{\TMV} \\
        \notMatchedArrow{\TMV} \\
        \uncover<4->{\ctxAnaFixedInto{\ctx}{\EMV_2}{\ECMV_2}{\TUnknown}}
      }{
        \ctxSynFixedInto{\ctx}{\EAp{\EMV_1}{\EMV_2}}{\uncover<5->{\ECApSynNonMatchedArrow{\ECMV_1}{\ECMV_2}}}{\uncover<6->{\TUnknown}}
      }
    }
  \end{mathpar}
  }
\end{frame}

\begin{frame}{The marked lambda calculus, altogether}
  \begin{center}
    \begin{tikzpicture}[box/.style={align=left, fill=white}]
      \Alt<4->{
        \node[box] (unmarked) {
          \begin{minipage}[t][1.5cm]{4cm}
            \emph{unmarked language} \\
            $\ctxSynTypeU{\ctx}{\EMV}{\TMV}$ \\
            $\ctxAnaTypeU{\ctx}{\EMV}{\TMV}$
          \end{minipage}
        };
      }{
        \node[box] (unmarked) {
          \begin{minipage}[t][1.5cm]{4cm}
            \emph{original language} \\
            $\ctxSyn{\ctx}{\EMV}{\TMV}$ \\
            $\ctxAna{\ctx}{\EMV}{\TMV}$
          \end{minipage}
        };
      }

      \node[box, visible on=<2->] (marking) [right=2.5cm of unmarked] {
        \begin{minipage}[t][1.5cm]{4cm}
          \emph{marking} \\
          $\ctxSynFixedInto{\ctx}{\EMV}{\ECMV}{\TMV}$ \\
          $\ctxAnaFixedInto{\ctx}{\EMV}{\ECMV}{\TMV}$
        \end{minipage}
      };

      \node[box, visible on=<3->] (marked) [below=2.5cm of marking] {
        \begin{minipage}[t][1.5cm]{4cm}
          \emph{marked language} \\
          $\ctxSynTypeM{\ctx}{\ECMV}{\TMV}$ \\
          $\ctxAnaTypeM{\ctx}{\ECMV}{\TMV}$
        \end{minipage}
      };

      \node[box, visible on=<5->] (erasure) [below=2.5cm of unmarked] {
        \begin{minipage}[t][1.5cm]{4cm}
          \emph{mark erasure} \\
          $\erase{\ECMV}$
        \end{minipage}
      };

      \draw[-latex, dashed, shorten >=5pt, visible on=<2->] (unmarked) -- (marking) node[midway, above] () {\small \emph{derive}};
      \draw[-latex, shorten >=5pt, visible on=<3->] (unmarked) to[bend left=20] (marked);
      \draw[-latex, shorten <=5pt, visible on=<5->] (marked)   to[bend left=20] (unmarked);
    \end{tikzpicture}
  \end{center}
\end{frame}

\begin{frame}[fragile]
  \begin{emphbox}{Theorem 2.2. Marking Well-Formedness}
    If $\ctxSynFixedInto{\ctx}{\EMV}{\ECMV}{\TMV}$,
      then $\ctxSynTypeM{\ctx}{\ECMV}{\TMV}$
        and $\erasesTo{\ECMV}{\EMV}$. \\
    If $\ctxAnaFixedInto{\ctx}{\EMV}{\ECMV}{\TMV}$,
      then $\ctxAnaTypeM{\ctx}{\ECMV}{\TMV}$
        and $\erasesTo{\ECMV}{\EMV}$.
  \end{emphbox}

  \pause
  \begin{emphbox}{Theorem 2.3(1). Marking Well-Typed Terms}
    If $\ctxSynTypeU{\ctx}{\EMV}{\TMV}$,
      then $\exists \ECMV$
        s.t. $\ctxSynFixedInto{\ctx}{\EMV}{\ECMV}{\TMV}$
        and $\markless{\ECMV}$. \\
    If $\ctxAnaTypeU{\ctx}{\EMV}{\TMV}$,
      then $\exists \ECMV$
        s.t. $\ctxAnaFixedInto{\ctx}{\EMV}{\ECMV}{\TMV}$
        and $\markless{\ECMV}$.
  \end{emphbox}
\end{frame}

\begin{frame}[fragile]{The recipe}
  \begin{itemize}
    \item Begin with a bidirectional gradually typed language

      \pause
    \item Derive marking rules from each typing rule
      %
      \begin{itemize}
        \item Consider the ``success'' case
        \item Consider the ``failure'' cases, introducing error marks
      \end{itemize}

      \pause
    \item \emph{Not} prescriptive \wrt{} localization strategy,\pause \eg,
      %
      \begin{itemize}
        \item use different marks, localize somewhere else, etc.
          \pause
        \item extend the judgments to accumulate facts that inform localization decisions
      \end{itemize}
  \end{itemize}
\end{frame}

\begin{frame}[fragile]{Marking local inconsistencies}
  The standard subsumption principle:
  \[%
    \inferrule[UASubsume]{
      \ctxSynTypeU{\ctx}{\EMV}{\TMV'} \\
      \consistent{\TMV}{\TMV'}
    }{
      \ctxAnaTypeU{\ctx}{\EMV}{\TMV}
    }
  \]%

  \pause
  becomes the analytic marking and (marked) typing rules:
  %
  \begin{mathpar}
    \inferrule[MKASubsume]{
      \ctxSynFixedInto{\ctx}{\EMV}{\ECMV}{\TMV'} \\
      \consistent{\TMV}{\TMV'}
    }{
      \ctxAnaFixedInto{\ctx}{\EMV}{\ECMV}{\TMV}
    }

    % \inferrule[MASubsume]{
      % \ctxSynTypeM{\ctx}{\ECMV}{\TMV'} \\
      % \consistent{\TMV}{\TMV'}
    % }{
      % \ctxAnaTypeM{\ctx}{\ECMV}{\TMV}
    % }
  \end{mathpar}
\end{frame}

\begin{frame}
  What if $\inconsistent{\TMV}{\TMV'}$, \eg, in $\ELam{f}{\TArrow{\TNum}{\TNum}}{\EAp{f}{\goodcolor{Red}{\ETrue}}}$?

  \pause
  \[%
    \ECMV \Coloneqq \cdots \mid \ECFree{x} \mid \ECApSynNonMatchedArrow{\ECMV}{\ECMV} \mid \colorbox{highlight}{\ECInconType{\ECMV}}
  \]%

  \pause
  \begin{mathpar}
    \inferrule[MKASubsume]{
      \ctxSynFixedInto{\ctx}{\EMV}{\ECMV}{\TMV'} \\
      \consistent{\TMV}{\TMV'}
    }{
      \ctxAnaFixedInto{\ctx}{\EMV}{\ECMV}{\TMV}
    }

    \inferrule[MKAInconsistentTypes]{
      \ctxSynFixedInto{\ctx}{\EMV}{\ECMV}{\TMV'} \\
      \inconsistent{\TMV}{\TMV'}
    }{
      \ctxAnaFixedInto{\ctx}{\EMV}{\ECInconType{\ECMV}}{\TMV}
    }

    % \inferrule[MAInconsistentTypes]{
      % \ctxSynTypeM{\ctx}{\ECMV}{\TMV'} \\
      % \inconsistent{\TMV}{\TMV'}
    % }{
      % \ctxAnaTypeM{\ctx}{\ECInconType{\ECMV}}{\TMV}
    % }
  \end{mathpar}
\end{frame}
