\newcommand{\entails}{\ensuremath{\goodcolor{\colorOkSideJudge}{\vdash}}}
\newcommand{\syn}{\ensuremath{\goodcolor{\colorOkSideJudge}{\Rightarrow}}}
\newcommand{\ana}{\ensuremath{\goodcolor{\colorOkSideJudge}{\Leftarrow}}}

\newcommand{\ctxSyn}[3]{\ensuremath{#1 \entails #2 \syn #3}}
\newcommand{\ctxAna}[3]{\ensuremath{#1 \entails #2 \ana #3}}

\begin{frame}[fragile]{Marked lambda calculus: a tutorial}
  \textbf{Start:} a small gradually typed lambda calculus*
  %
  \[\begin{array}{rcl}
    \TMV  & \Coloneqq & \TUnknown \mid \TNum \mid \TBool \mid \TArrow{\TMV}{\TMV} \\
    \EMV  & \Coloneqq & x \mid \ELam{x}{\TMV}{\EMV} \mid \EAp{\EMV}{\EMV}
    \mid                   \ENumMV \mid \ETrue \mid \EFalse \mid \EIf{\EMV}{\EMV}{\EMV}
  \end{array}\]

  \pause

  \hphantom{\textbf{Start:} }with a standard bidirectional type system
  %
  \[\begin{array}{rl}
    \ctxSyn{\ctx}{\EMV}{\TMV} & \text{\small ($\EMV$ synthesizes type $\TMV$)} \\
    \ctxAna{\ctx}{\EMV}{\TMV} & \text{\small ($\EMV$ analyzes against type $\TMV$)}
  \end{array}\]
  \pause
  *We need only the \emph{static semantics} for ill-typed programs!

  \note[item]{For the purposes of this short tutorial, we'll start with a small gradually typed lambda
    calculus}
  \note[item]{with a standard bidirectional type system}
  \note[item]{where synthesis corresponds to inferring the type of a term from its syntax}
  \note[item]{and analysis is checking a term against an expected type.}
  \note[item]{Now, we have here the unknown type, but in fact it's necessary as we'll see for
    ill-typed programs only, and we require then only the static semantics of the gradually typed
    lambda calculus}
\end{frame}

\begin{frame}[fragile]{Typing variable occurrences}
  Synthesizing the type of a variable is standard:
  \[%
    \inferrule[SVar]{
      \uncover<2->{\samesizecolorbox{highlight}{\inCtx{\ctx}{x}{\TMV}}}
    }{
      \ctxSyn{\ctx}{x}{\TMV}
    }
  \]%

  \uncover<3->{
    But what if $\notInCtx{\ctx}{x}$?
  }

  \note[item]{In our standard bidirectional type system, we would probably have a rule that looks
    like this incomplete one for synthesizing the type of a variable occurrence}
  \note[item]{We can conclude, of course, that x synthesizes tau if we have the pair in the context}
  \note[item]{OK, that's all well and good, but---and this is where totality comes in---what if x
    isn't in scope?}
\end{frame}

\begin{frame}[fragile]
  How do we handle this failure case?

  \begin{lstlisting}[style=OCaml, escapechar=@]
let rec syn ctx e =
  match e with
    Var x ->
      match Ctx.lookup ctx x with
        Some ty -> @\only<1-2>{ty}\only<3>{\goodcolor{PineGreen}{Ok(ty)}}@
        None    -> @\goodcolor{red}{\only<1>{???}\only<2>{failwith (x ++ " is unbound")}\only<3>{Error(UnboundError(...))}}@
    ...
  \end{lstlisting}

  \note[item]{This failure case poses a problem if we were to try implementing this rule, which like
    look something like this OCaml code}
  \note[item]{The None case of the variable lookup corresponds exactly to being unable to satisfy
    the premise, and it's not immediately obvious what we should do}
  \note[item]{A common approach is to simply throw an exception with a diagnostic and leave it to
    the caller to handle}
  \note[item]{or similarly by returning an error value}
  \note[item]{In either case, we've aborted the type-checking early at the first error, and this
    failure to recover just isn't going to cut it for us 21st century folk}
\end{frame}

\begin{frame}[fragile]
  We've \emph{localized} the error: ``this occurrence of $x$ is unbound!'' \\[1em]

  \pause
  How can we \emph{recover}? \\[1em]

  \pause
  \textbf{Solution.} $\TUnknown$ \pause \hspace{1em} $\leftarrow$ unknown type

  \note[item]{At this point, we've successfully localized the error}
  \note[item]{But how can we recover?}
  \note[item]{Given what I said early, and the fact that this session is all about gradual typing,
    you've probably guessed it by now, but we'll do so via the unknown type}
\end{frame}

\begin{frame}[fragile]{From type checking to marking}
  \textbf{Idea.} Augment the type checking process with \emph{marking}! \\[1em]

  \pause
  \begin{itemize}
    \item localize and report the error as a \emph{mark}
      \pause
      \begin{itemize}
        \item compiler messages

          \pause
        \item editor decorations
      \end{itemize}

      \pause
    \item use the unknown type to encapsulate missing type information
  \end{itemize}

  \begin{mathpar}
    \inferrule{ }{
      \consistent\TUnknown{\TMV}
    }

    \inferrule{ }{
      \consistent{\TMV}{\TUnknown}
    }

    \inferrule{ }{
      \consistent{\TMV}{\TMV}
    }

    \inferrule{
      \consistent{\TMV_1}{\TMV_1'} \\
      \consistent{\TMV_2}{\TMV_2'} \\
    }{
      \consistent{\TArrow{\TMV_1}{\TMV_2}}{\TArrow{\TMV_1'}{\TMV_2'}}
    }
  \end{mathpar}

  \note[item]{The idea is to augment the type checking process with marking}
  \note[item]{where we localize and report errors by adding marks}
  \note[item]{which intuitively correspond to compiler diagnostics or editor decorations}
  \note[item]{Then, we'll use the unknown type, which provides exactly the right mechanisms via
    the consistency relation to recover in a principled manner}
\end{frame}

\begin{frame}[fragile]{A two-layer calculus}
  Supplement the \emph{unmarked} syntax
  %
  \[\begin{array}{rcl}
    \TMV  & \Coloneqq & \TUnknown \mid \TNum \mid \TBool \mid \TArrow{\TMV}{\TMV} \\
    \EMV  & \Coloneqq & x \mid \ELam{x}{\TMV}{\EMV} \mid \EAp{\EMV}{\EMV}
            \mid           \ENumMV \mid \ETrue \mid \EFalse \mid \EIf{\EMV}{\EMV}{\EMV}
  \end{array}\]

  \pause
  into a \emph{marked} one that contains \emph{error marks}:
  \[\begin{array}{rcl}
    \ECMV & \Coloneqq & x \mid \ECLam{x}{\TMV}{\ECMV} \mid \ECAp{\ECMV}{\ECMV}
            \mid           \ENumMV \mid \ECTrue \mid \ECFalse \mid \ECIf{\ECMV}{\ECMV}{\ECMV}
            \mid           \samesizecolorbox{highlight}{\ECFree{x}}
  \end{array}\]

  $\ECFree{x}$ is a \emph{marked term} denoting a free occurrence of $x$

  \note[item]{We'll take the original syntax, which we call unmarked, and add the aforementioned
    error marks, which are given by fancy parentheses and a subscript in red, but they're just
    symbolic representations of actual error messages}
  \note[item]{For now, we've only a single error mark that denotes exactly the variable is free}
\end{frame}

\begin{frame}[fragile]
  Extend the original typing judgments
  \[\begin{array}{rl}
    \ctxSyn{\ctx}{\EMV}{\TMV} & \text{\small ($\EMV$ synthesizes type $\TMV$)} \\
    \ctxAna{\ctx}{\EMV}{\TMV} & \text{\small ($\EMV$ analyzes against type $\TMV$)}
  \end{array}\]

  \pause
  into the bidirectional \textbf{marking judgments}:
  \[\begin{array}{rl}
    \ctxSynFixedInto{\ctx}{\EMV}{\ECMV}{\TMV} & \text{{\small ($\EMV$ is marked into $\ECMV$ and synthesizes type $\TMV$)}} \\
    \ctxAnaFixedInto{\ctx}{\EMV}{\ECMV}{\TMV} & \text{{\small ($\EMV$ is marked into $\ECMV$ and analyzes against type $\TMV$)}}
  \end{array}\]

  \pause
  Type-based semantic services use marked terms!
  \[\begin{array}{rl}
    \ctxSynTypeM{\ctx}{\ECMV}{\TMV} & \text{\small ($\ECMV$ synthesizes type $\TMV$)} \\
    \ctxAnaTypeM{\ctx}{\ECMV}{\TMV} & \text{\small ($\ECMV$ analyzes against type $\TMV$)}
  \end{array}\]

  \note[item]{Now, we'll extend the original typing judgments into marking judgments}
  \note[item]{where unmarked terms are marked into marked terms}
  \note[item]{These now are the judgments that express the error-localizing and recovering
    type-checking process}
  \note[item]{Downstream services dependent on type information will use these typed marked terms}
\end{frame}

\begin{frame}[fragile]{Marking free variables}
  One typing rule for variables 
  \[%
    \inferrule[SVar]{
      \inCtx{\ctx}{x}{\TMV}
    }{
      \ctxSyn{\ctx}{x}{\TMV}
    }
  \]%

  becomes two synthetic marking rules: 
  \begin{mathpar}
    \uncover<2->{
      \inferrule[MKSVar]{
        \inCtx{\ctx}{x}{\TMV}
      }{
        \ctxSynFixedInto{\ctx}{x}{x}{\TMV}
      }
    }

    \uncover<3->{
      \inferrule[MKSFree]{
        \notInCtx{\ctx}{x}
      }{
        \uncover<3->{\ctxSynFixedInto{\ctx}{x}{\uncover<4->{\ECFree{x}}}{\uncover<5->{\TUnknown}}}
      }
    }
  \end{mathpar}

  \note[item]{Now, thinking about both the success and failure cases, our synthetic typing rule for
    variables becomes two synthetic marking rules}
  \note[item]{In the success case we don't need to mark any errors}
  \note[item]{And in the failure case we'll mark x with the free variable mark and synthesize the
    unknown type, since we don't know anything about it}
\end{frame}

\begin{frame}[fragile]{Marking local inconsistencies}
  The standard subsumption principle:
  \[%
    \inferrule[UASubsume]{
      \ctxSyn{\ctx}{\EMV}{\TMV'} \\
      \consistent{\TMV}{\TMV'}
    }{
      \ctxAna{\ctx}{\EMV}{\TMV}
    }
  \]%

  \pause
  becomes the analytic marking rule:
  %
  \begin{mathpar}
    \inferrule[MKASubsume]{
      \ctxSynFixedInto{\ctx}{\EMV}{\ECMV}{\TMV'} \\
      \consistent{\TMV}{\TMV'}
    }{
      \ctxAnaFixedInto{\ctx}{\EMV}{\ECMV}{\TMV}
    }
  \end{mathpar}

  \note[item]{Another common error occurs when some term's type doesn't match the type that's
    expected}
  \note[item]{In a bidirectional system, this would occur most commonly when trying to apply a
    subsumption principle like the one shown}
  \note[item]{We can straightforwardly derive a similar-looking marking rule}
\end{frame}

\begin{frame}
  What if $\inconsistent{\TMV}{\TMV'}$, \eg, in $\EPlus{\ENum{5}}{\alt<4->{\fboxsep=1pt\samesizecolorbox{highlight}{\ECInconType{\ETrue}}}{\ETrue}}$?

  \pause
  \[%
    \ECMV \Coloneqq \cdots \mid \ECFree{x} \mid \samesizecolorbox{highlight}{\ECInconType{\ECMV}}
  \]%

  \pause
  \begin{mathpar}
    \inferrule[MKASubsume]{
      \ctxSynFixedInto{\ctx}{\EMV}{\ECMV}{\TMV'} \\
      \consistent{\TMV}{\TMV'}
    }{
      \ctxAnaFixedInto{\ctx}{\EMV}{\ECMV}{\TMV}
    }

    \inferrule[MKAInconsistentTypes]{
      \ctxSynFixedInto{\ctx}{\EMV}{\ECMV}{\TMV'} \\
      \inconsistent{\TMV}{\TMV'}
    }{
      \ctxAnaFixedInto{\ctx}{\EMV}{\ECInconType{\ECMV}}{\TMV}
    }
  \end{mathpar}

  \note[item]{That's all well and good, but, totality guiding us again, we consider the failure case
    of when the types aren't consistent}
  \note[item]{And, just as last time, we add a new mark denoting this type error}
  \note[item]{Then, it's again straightforward to derive the second rule}
  \note[item]{which would mark true in this example like so}
  \note[item]{Hopefully, these two examples gives you a sense of how this framework works}
  \note[item]{To continue the development for our minimal language, we would simply follow this pattern}
\end{frame}

\begin{frame}{\ldots\ and the rest}
  \[\begin{array}{rcll}
    \EMV  & \Coloneqq & \cdots \\
          & \mid         & \ECApSynNonMatchedArrow{\ECMV_1}{\ECMV_2}
                      & \textup{\small $\ECMV_1$ synthesizes non-matched arrow type} \\
          & \mid         & \ECInconBr{\ECMV_1}{\ECMV_2}{\ECMV_3}
                      & \textup{\small branches synthesize inconsistent types} \\
          & \mid         & \ECLamAnaNonMatchedArrow{x}{\TMV}{\ECMV}
                      & \text{\small analysis against non-matched arrow type} \\
          & \mid         & \ECLamInconAsc{x}{\TMV}{\ECMV}
                      & \textup{\small ascription inconsistent with domain} \\
  \end{array}\]

  \note[item]{And it turns out you would need this set of marks, which cover the remaining failure
    cases that would arise during type checking}
  \note[item]{For the sake of time, I won't cover them---you can find their full development in the
    paper}
  \note[item]{With this development in place, we should be wondering now, how we know this recipe
    has given us the right rules and that we haven't made a mistake?}
\end{frame}

\begin{frame}[fragile]{A total marking}
  \begin{emphbox}{Totality}
    These semantics should give meaning to \emph{all well-typed and ill-typed programs}.
  \end{emphbox}
  \vspace{-1em}
  \pause
  \begin{center}
    $\bm{\Downarrow}$
  \end{center}
  \vspace{-1em}
  \begin{emphbox}{Theorem 2.1. Totality}
    For all $\ctx$ and $\EMV$,
      $\exists \ECMV, \TMV$
        s.t. $\ctxSynFixedInto{\ctx}{\EMV}{\ECMV}{\TMV}$. \\
    For all $\ctx$, $\EMV$, and $\TMV$,
      $\exists \ECMV$
        s.t. $\ctxAnaFixedInto{\ctx}{\EMV}{\ECMV}{\TMV}$. \\
    \textcolor{Black!50}{(Every unmarked term can be marked under any context!)}
  \end{emphbox}

  \note[item]{From the start I mentioned that a notion of totality would guide us, and now we can
    formally state it to ensure we haven't missed any rules}
  \note[item]{The theorem states that for every unmarked expression, we should be able to mark it
    both synthetically and analytically}
  \note[item]{Of course, there's an accompanying theorem that requires that marking gives only one
    result, as one would hope}
\end{frame}

\begin{frame}[fragile]{A sensible marking}
  \begin{emphbox}{Theorem 2.2. Well-Formedness}
    If $\ctxSynFixedInto{\ctx}{\EMV}{\ECMV}{\TMV}$,
      then $\ctxSynTypeM{\ctx}{\ECMV}{\TMV}$
        and $\alt<2->{\fboxsep=1pt\samesizecolorbox{highlight}{\erasesTo{\ECMV}{\EMV}}}{\erasesTo{\ECMV}{\EMV}}$. \\
    If $\ctxAnaFixedInto{\ctx}{\EMV}{\ECMV}{\TMV}$,
      then $\ctxAnaTypeM{\ctx}{\ECMV}{\TMV}$
        and $\alt<2->{\fboxsep=1pt\samesizecolorbox{highlight}{\erasesTo{\ECMV}{\EMV}}}{\erasesTo{\ECMV}{\EMV}}$. \\
    \textcolor{Black!50}{(Marking only adds marks, \ie, marking then erasing is identity!)}
  \end{emphbox}

  \uncover<3->{
  \begin{emphbox}{Theorem 2.3(1). Well-Typed Terms}
    If $\ctxSynTypeU{\ctx}{\EMV}{\TMV}$,
      then $\exists \ECMV$
        s.t. $\ctxSynFixedInto{\ctx}{\EMV}{\ECMV}{\TMV}$
        and $\alt<4->{\fboxsep=1pt\samesizecolorbox{highlight}{\markless{\ECMV}}}{\markless{\ECMV}}$. \\
    If $\ctxAnaTypeU{\ctx}{\EMV}{\TMV}$,
      then $\exists \ECMV$
        s.t. $\ctxAnaFixedInto{\ctx}{\EMV}{\ECMV}{\TMV}$
        and $\alt<4->{\fboxsep=1pt\samesizecolorbox{highlight}{\markless{\ECMV}}}{\markless{\ECMV}}$.
    \textcolor{Black!50}{(Marking well-typed terms introduces no marks!)}
  \end{emphbox}
  }

  \note[item]{Secondly, marking must not deform the syntax except by adding marks, and removing
  these marks, which is given by this square superscript erasure operation, should give us back what we started with}
  \note[item]{We'd would also like that marking doesn't introduce any error marks on
    well-typed terms, which we state by a markless judgment}
  \note[item]{There'd be a corresponding statement for ill-typed terms, which should have at least
    one mark}
\end{frame}

\begin{frame}[fragile]
  \vspace{-1em}
  % https://q.uiver.app/#q=WzAsMyxbMCwwLCJlIl0sWzIsMiwiXFxHYW1tYSBcXHZkYXNoIFxcY2hlY2t7ZX0gXFxSaWdodGFycm93IFxcdGF1IFxcXFwgXFxHYW1tYSBcXHZkYXNoIFxcY2hlY2t7ZX0gXFxMZWZ0YXJyb3cgXFx0YXUiXSxbMiwwLCJcXEdhbW1hIFxcdmRhc2ggZSBcXGxvb3BhcnJvd3JpZ2h0IFxcY2hlY2t7ZX0gXFxSaWdodGFycm93IFxcdGF1IFxcXFwgXFxHYW1tYSBcXHZkYXNoIGUgXFxsb29wYXJyb3dyaWdodCBcXGNoZWNre2V9IFxcTGVmdGFycm93IFxcdGF1Il0sWzAsMSwiIiwwLHsiY3VydmUiOi0zLCJjb2xvdXIiOlszMCw2MCw2MF19XSxbMCwyLCIiLDAseyJzdHlsZSI6eyJib2R5Ijp7Im5hbWUiOiJkb3R0ZWQifX19XSxbMSwwLCJ7XFxffV57XFxzcXVhcmV9IiwwLHsiY3VydmUiOi0zLCJjb2xvdXIiOlswLDAsMjBdfSxbMCwwLDIwLDFdXSxbMywyLCIiLDAseyJzaG9ydGVuIjp7InNvdXJjZSI6MjB9LCJjb2xvdXIiOlswLDAsNDVdfV1d
  \newcommand{\mlnode}[1]{\begin{tabular}{@{}c@{}}#1\end{tabular}}
  \[\begin{tikzcd}[arrow style=tikz]
      e
        & & {\mlnode{
              \text{\textcolor{Black!50}{\small (\emph{marking})}} \\
              $\ctxSynFixedInto{\ctx}{\EMV}{\ECMV}{\TMV}$ \\
              $\ctxAnaFixedInto{\ctx}{\EMV}{\ECMV}{\TMV}$
            }} \\
      \\
        & & {\mlnode{
              $\ctxSynTypeM{\ctx}{\ECMV}{\TMV}$ \\
              $\ctxAnaTypeM{\ctx}{\ECMV}{\TMV}$ \\
              \text{\textcolor{Black!50}{\small (\emph{marked})}}}}
          & {\text{\textcolor{Black!50}{\small (downstream services)}}}
    \arrow["{}"{name=0, anchor=center, inner sep=0}, color={\colorMKText}, curve={height=-40pt}, from=1-1, to=3-3]
    \arrow["{\text{\textcolor{Black!50}{\small (\emph{erasure})} } \erase{\_}}", curve={height=-40pt}, from=3-3, to=1-1]
    \arrow["{}", dashed, color={Black!50}, from=3-3, to=3-4]
  \end{tikzcd}\]

  \vspace{-1em}
  \pause
  \begin{center}
    \small
    Metatheory completely mechanized in Agda
    \textcolor{MidnightBlue}{[\href{https://github.com/hazelgrove/error-localization-agda}{hazelgrove/error-localization-agda}]} \ 
  \end{center}

  \note[item]{Putting these pieces together, we can view marking as a total function that
    takes arbitrary terms of the unmarked syntax, well-typed or not, to well-typed terms of the marked
    syntax}
  \note[item]{These are then used by downstream services like code completion}
  \note[item]{The well-formedness theorem is then described by this identity loop}
  \note[item]{All this metatheory, mechanized in Agda, can be found in our artifact}
\end{frame}

\begin{frame}[fragile]{The Recipe}
  \begin{itemize}
    \item Begin with a bidirectional gradually* typed language
      %
      \begin{itemize}
        \item *Only need the static parts for marking ill-typed programs!
      \end{itemize}


      \pause
    \item Derive marking rules from each typing rule
      %
      \begin{itemize}
        \item Consider the ``success'' case
        \item Consider the ``failure'' cases, introducing error marks
      \end{itemize}

      \pause
    \item \emph{Not} prescriptive \wrt{} localization strategy
      %
      \begin{itemize}
        \item We formalize three possible localization strategies for if-then-else with inconsistent
          branches
      \end{itemize}
  \end{itemize}

  \note[item]{So, we've described this recipe by which you can also take your existing type system
    and formalize how type errors are localized and recovered from}
  \note[item]{You could technically start with a non-gradual type system, but you'll need to add the
    unknown type for marking ill-typed programs}
  \note[item]{Systematically considering the success and failure cases of each typing rule, we saw
    how we can derive the marking rules}
  \note[item]{And all the decisions about how to actually localize and mark were made along the way
    by us, the language designers}
  \note[item]{The framework doesn't prescribe a particular localization strategy; we make no claims
    about having the only or best way to localize errors}
  \note[item]{We're just giving the judgmental tools to formally describe how you're localizing}
  \note[item]{In the paper, we actually formalize the various localization strategies we saw for
    conditionals at the beginning of the talk}
\end{frame}
