\begin{frame}{Bidirectional and unification, together}
  The two systems supplement each other: \\[1em]

  \pause
  \begin{itemize}
    \item \emph{marked lambda calculus}: \\
      a general recipe for formally describing local type error localization and recovery

      \pause
    \item \emph{type hole inference}: \\
      a workflow for handling globally inconsistent constraints
  \end{itemize}

  \vspace{1em}
  \pause
  Consider using these techniques for your next language!

  \note[item]{Hopefully, we've seen how the bidirectional and unification approaches are not
  necessarily opposing or alternatives, but can be elegantly blended}
  \note[item]{To review, our contributions of the marked lambda calculus gives a general recipe for
    formally specifying typing semantics that localize and recover from local type errors}
  \note[item]{and type hole inference gives a workflow to handle these global inconsistencies}

  \note[item]{Consider using these techniques when designing your next language, or tooling or
    editor support for your next language}
\end{frame}

\begin{frame}{More in the paper and artifact}
  \begin{itemize}
    \item A full description of the marked lambda calculus\pause, and

      \begin{itemize}
        \item full metatheory\pause, also mechanized
          \textcolor{MidnightBlue}{[\href{https://github.com/hazelgrove/error-localization-agda}{hazelgrove/error-localization-agda}]}

          \pause
        \item extensions to richer typing features \\ \pause
          (parametric polymorphism and destructuring let)

          \pause
        \item connections to structured editing
      \end{itemize}

      \pause
    \item A more thorough discussion of type hole inference\pause, and

      \begin{itemize}
        \item filling expression holes

          \pause
        \item polymorphic generalization
      \end{itemize}

      \pause
    \item Implementations of both in Hazel
      \textcolor{BrickRed}{\small[\href{https://hazel.org}{hazel.org}]}
      \textcolor{MidnightBlue}{\small[\href{https://github.com/hazelgrove/hazel}{hazelgrove/hazel}]}
  \end{itemize}

  \note[item]{There is, of course, more in the paper}
  \note[item]{We give a complete development of the marked lambda calculus for the language of this
    presentation additionally with products}
  \note[item]{and provide the full metatheory}
  \note[item]{which is also mechanized in Agda}
  \note[item]{We explore extensions to richer typing features, specifically developing semantics for
  System F-style parametric polymorphism and destructuring lets}
  \note[item]{We also discuss connections to structured editing, specifically the Hazelnut
  structured editing calculus on which Hazel is based}
  \note[item]{The paper also includes a more thorough discussion of type hole inference that we've
  brushed over today}
  \note[item]{We discuss further expansions of the workflow when empty expression holes are added}
  \note[item]{and how polymorphic generalization would work in this system}

  \note[item]{And, we have implementations of both in Hazel}
  \note[item]{Thanks!}
\end{frame}
