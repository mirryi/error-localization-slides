\begin{frame}[fragile]{More in the paper and artifact}
  \begin{tikzpicture}[remember picture,overlay]
    \node[xshift=-1.3cm,yshift=-1.3cm] at (current page.north east)
      {\includegraphics[width=1.5cm]{media/artifact-available.png}};
    \node[xshift=-1.3cm,yshift=-3cm] at (current page.north east)
      {\includegraphics[width=1.5cm]{media/artifact-reusable.png}};
  \end{tikzpicture}
  
  \vspace{-1em}
  \begin{itemize}
    \setlength\itemsep{1.5ex}
    \item A full description of the marked lambda calculus\pause, and

      \begin{itemize}
        \item mechanization in Agda
          \textcolor{MidnightBlue}{[\href{https://github.com/hazelgrove/error-localization-agda}{hazelgrove/error-localization-agda}]} \ 

          \pause
        \item extensions to richer typing features \\ \pause
          (parametric polymorphism and destructuring let)

          \pause
        \item connections to structured editing
      \end{itemize}

      \pause
    \item A more thorough discussion of type hole inference\pause, and

      \begin{itemize}
        \item filling expression holes

          \pause
        \item polymorphic generalization
      \end{itemize}

      \pause
    \item Implementations of both in Hazel
      \textcolor{BrickRed}{\small[\href{https://hazel.org}{hazel.org}]}
      \textcolor{MidnightBlue}{\small[\href{https://github.com/hazelgrove/hazel}{hazelgrove/hazel}]}
  \end{itemize}

  \pause
  \vspace{1em}
  Consider using these techniques for your next language!

  \note[item]{There is, of course, more in the paper and our artifact, which you should check out}
  \note[item]{Please, consider trying these techniques when designing your next language, or tooling
    or editor support for your next language!}
  \note[item]{If nothing else, we hope this work spurs more work on the semantics of ill-typed
    programs}
  \note[item]{Thanks!}
\end{frame}
