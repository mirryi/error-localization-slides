\begin{frame}{Bidirectional and unification, together}
  The two systems supplement each other: \\[1em]

  \pause
  \begin{itemize}
    \item \emph{marked lambda calculus}: \\
      a general recipe for local type error localization and recovery

      \pause
    \item \emph{type hole inference}: \\
      a workflow for handling globally inconsistent constraints
  \end{itemize}

  \vspace{1em}
  \pause
  Consider using these techniques for your next language!
\end{frame}

\begin{frame}{More in the paper and artifact}
  \begin{itemize}
    \item A full description of the marked lambda calculus\pause, and

      \begin{itemize}
        \item full metatheory\pause, also mechanized
          \textcolor{MidnightBlue}{[\href{https://github.com/hazelgrove/error-localization-agda}{hazelgrove/error-localization-agda}]}

          \pause
        \item extensions to richer typing features \\ \pause
          (parametric polymorphism and destructuring let)

          \pause
        \item connections to structured editing
      \end{itemize}

      \pause
    \item A more thorough discussion of type hole inference\pause, and

      \begin{itemize}
        \item filling expression holes

          \pause
        \item polymorphic generalization
      \end{itemize}

      \pause
    \item Implementations of both in Hazel
      \textcolor{BrickRed}{\small[\href{https://hazel.org}{hazel.org}]}
      \textcolor{MidnightBlue}{\small[\href{https://github.com/hazelgrove/hazel}{hazelgrove/hazel}]}
  \end{itemize}
\end{frame}
